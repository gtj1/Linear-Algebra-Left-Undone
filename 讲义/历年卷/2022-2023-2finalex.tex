\phantomsection
\section*{2022-2023学年线性代数II(H)期末考前练习}
\addcontentsline{toc}{section}{2022-2023学年线性代数II(H)期末考前练习}

\begin{center}
    任课老师:刘康生\hspace{4em} 考试时长:无
\end{center}

\begin{enumerate}
	\item[一、]$A$是复(实)正规矩阵的充要条件是:存在复(实)矩阵$S_1,S_2$满足$\overline{S_1}^{\mathrm{T}}=S_1$,$\overline{S_2}^{\mathrm{T}}=-S_2$,$S_1S_2=S_2S_1$,$A=S_1+S_2$. 另外,$A$的如上分解是唯一的.
	\item[二、]证明:$A$是实正规矩阵的充要条件是存在镜面映像$\pi_1,\pi_2,\ldots,\pi_t$使得
    \[A^\mathrm{T}=\pi_t\cdots\pi_2\pi_1A.\]
	\item[三、]将如下$\mathbf{R}^3$上变换$T$表示为两个镜面映像之积:

    $T:$先绕$x$轴旋转$\varphi$角度,再绕$z$轴旋转$\theta$角度(右手系,$0<\varphi,\theta<\cfrac{\pi}{2}$).
	\item[四、]求下列变换的所有不变子空间:
	\begin{enumerate}[label=(\arabic*)]
        \item $\sigma_A\in\mathcal{L}(\mathbf{R}^2)$,$A=\begin{pmatrix}
            0 & 1 \\ a & 0
        \end{pmatrix}$;

        \item $T\in\mathcal{L}(V)$,$\dim V=n$,$T^n=O$,$T^{n-1}\neq O$;

        \item $\sigma\in\mathcal{L}(\mathbf{C}^5)$,$A=\begin{pmatrix}
            0 & 0 & 0 & 0 & 0 \\ 1 & 0 & 0 & 0 & 0 \\ 0 & 1 & 0 & 0 & 1 \\ 0 & 0 & 1 & 0 & -3 \\ 0 & 0 & 0 & 1 & 3
        \end{pmatrix}$,$\det(\lambda E-A)=\lambda^2(\lambda-1)^3$.
    \end{enumerate}
	\item[五、]设$A=\begin{pmatrix}
        0 & 20 & 23 & 0 \\ 0 & 0 & 6 & 28 \\ 0 & 0 & 0 & 0 \\ 0 & 0 & 0 & 0
    \end{pmatrix}$,证明:不存在复矩阵$B$使得$B^2=A$.
	\item[六、]已知$\alpha_1=(1,2,0)$,$\alpha_2=(0,1,2)$,$\alpha_3=(2,0,1)\in\mathbf{R}^3$. 设$T\in\mathcal{L}(\mathbf{R}^3)$,且有$T(\alpha_1)=(-1,0,-2)$,$T(\alpha_2)=(-2,-1,0)$,$T(\alpha_3)=(0,-2,-1)$. 证明:$T$是一个镜面映像.

    \item[七、]定义在 $ V = \mathbf{R}^3 $ 上的运算
    \[ \langle \boldsymbol{x}, \boldsymbol{y} \rangle_V = a(x_2-x_1)(y_2-y_1)+bx_2y_2+x_3y_3(a,b>0). \]
    其中 $ \boldsymbol{x} = (x_1, x_2, x_3) $,$ \boldsymbol{y} = (y_1, y_2, y_3) $.
    \begin{enumerate}[label=(\arabic*)]
        \item 验证 $ \langle \cdot, \cdot \rangle_V $ 是 $ \mathbf{R}^3 $ 上的一个内积;

        \item 求 $ \mathbf{R}^3 $ 在 $ \langle \cdot, \cdot \rangle_V $ 下的一组标准正交基;

        \item 求 $ \boldsymbol{\beta} \in V $ 使得 $ \forall \boldsymbol{x} \in V: x_1 + x_2 + x_3 = \langle \boldsymbol{x}, \boldsymbol{\beta} \rangle_V $.
    \end{enumerate}

\item[八、]考虑二直线
    \[l_1:\begin{cases}
        x=t \\ y=-t-1 \\ z=3t
    \end{cases},\enspace l_2:\begin{cases}
        ax+2y+z=0 \\ x-y-z+d=0,
    \end{cases}\]
    求$a$,$d$满足的条件,使得二直线
    \begin{enumerate}[label=(\arabic*)]
        \item 平行;

        \item 重合;

        \item 相交;

        \item 异面.
    \end{enumerate}
\end{enumerate}


\clearpage

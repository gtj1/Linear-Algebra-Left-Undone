\phantomsection
\section*{2021-2022学年线性代数I(H)期中}
\addcontentsline{toc}{section}{2021-2022学年线性代数I(H)期中(吴志祥老师)}

\begin{center}
    任课老师:吴志祥\hspace{4em} 考试时长:120分钟
\end{center}
\begin{enumerate}
	\item[一、](10分)$A=\begin{pmatrix}
        1 & 1 & 1 \\ 2 & 1 & -a \\ 1 & -2 & -3
    \end{pmatrix}$,$X=\begin{pmatrix}
        x_1 \\ x_2 \\ x_3
    \end{pmatrix}$,$b=\begin{pmatrix}
        3 \\ 9 \\ -6
    \end{pmatrix}$,且$AX=b$无解,求$a$.
	\item[二、](10分)证明替换定理:设向量组$\alpha_1,\ldots,\alpha_s$线性无关,$\beta=b_1\alpha_1+\cdots+b_s\alpha_s$. 如果$\beta_i\neq 0$,则$\beta$可替换$\alpha_1,\ldots,\alpha_s$中的某个向量成为一个新的线性无关向量组.
	\item[三、](10分)设$\{\varepsilon_1,\varepsilon_2,\varepsilon_3,\varepsilon_4,\varepsilon_5\}$是欧式空间$V$的一组标准正交基,$W=\spa(\alpha_1,\alpha_2,\alpha_3)$,其中$\alpha_1=2\varepsilon_1+\varepsilon_2+\varepsilon_3,\alpha_2=\varepsilon_1+\varepsilon_2+\varepsilon_5,\alpha_3=\varepsilon_1-\varepsilon_2+\varepsilon_4$.
	\begin{enumerate}[label=(\arabic*)]
        \item 求$\alpha_1,\alpha_2$的夹角;

        \item 求$W$的一组标准正交基.
    \end{enumerate}
	\item[四、](10分)证明:如果向量组$\{\alpha_1,\ldots,\alpha_m\}$的秩为$r$,那么该向量组中任意$s$个向量组成的子集的秩大于等于$r+s-m$.
	\item[五、](10分)在$\mathbf{R}^3$中取三个向量
	\[\alpha_1=(1,-2,0),\alpha_2=(-3,0,-2),\alpha_3=(2,4,3),\]
    设$\sigma$是满足$\sigma(\alpha_i)=\e_i(i=1,2,3)$的线性变换,其中$\{e_1,e_2,e_3\}$是$\mathbf{R}^3$的自然基.
    \begin{enumerate}[label=(\arabic*)]
        \item 求$\sigma$关于自然基所对应的矩阵;

        \item 求向量$\alpha_1=(-2,5,6)$在$\sigma$下的像.
    \end{enumerate}
	\item[六、] (10分)已知$\mathbf{R}[x]_n$的线性变换$\sigma$满足$\sigma(p(x))=p(x+1)-p(x),\enspace p(x)\in\mathbf{R}[x]_n$.
	\begin{enumerate}[label=(\arabic*)]
        \item 求$\sigma$的秩与核;

        \item 求所有可能的$p(x)\in\mathbf{R}[x]_n$和$\lambda\in\mathbf{R}$使得$\sigma(p(x))=\lambda p(x)$.
    \end{enumerate}
	\item[七、](10分)设$B=\{\alpha_1,\alpha_2,\alpha_3,\alpha_4\}$是4维线性空间$V$的一组基,$\sigma$关于基$B$的矩阵为
    \[A=\begin{pmatrix}1 & 0 & 2 & 1 \\ -1 & 2 & 1 & 3 \\ 1 & 2 & 5 & 5 \\ 2 & -2 & 1 & -2\end{pmatrix},\]
    求$\sigma$的像与核.

    \item[八、](10分)域$\mathbf{F}$上所有$m\times n$矩阵组成的集合$M_{m\times n}(\mathbf{F})$是域$\mathbf{F}$上的线性空间. 定义$V_i=\{Ae_{ii}\mid A\in M_{m\times n}(\mathbf{F})\}(i=1,2,\ldots,n)$,其中$e_{ij}$是第$i$行第$j$列元素为1,其余元素均为0的$n$阶矩阵,证明:
    \begin{enumerate}[label=(\arabic*)]
        \item $V_i$是$M_{m\times n}(\mathbf{F})$的子空间;

        \item $M_{m\times n}(\mathbf{F})=V_1\oplus V_2\oplus\cdots\oplus V_n$.
    \end{enumerate}
	\item[九、](20分)判断下列命题的真伪,若它是真命题,请给出简单的证明;若它是伪命题,给出理由或举反例将它否定.
    \begin{enumerate}[label=(\arabic*)]
        \item 正整数集$\mathbf{R}^+$对如下定义的加法和数量乘法构成整数$\mathbf{Z}$上的线性空间:
        \[a\oplus b=ab,\enspace\lambda\circ a=a^\lambda,\enspace\forall a,b\in\mathbf{R}^+,\lambda\in\mathbf{Z};\]

        \item 设$\sigma\in\mathcal{L}(V)$,$\{\alpha_1,\ldots,\alpha_n\}$是$V$的一组基,则$\sigma$可逆当且仅当$\{\sigma(\alpha_1),\ldots,\sigma(\alpha_n)\}$是$V$的一组基;

        \item 对任意实数域$\mathbf{R}$上线性空间$V$,都能找到有限个$V$的非平凡子空间$V_1,\ldots,V_m$使得$V_1\cup\cdots\cup V_m$;

        \item 与所有$n$阶矩阵可交换的矩阵一定是$n$阶数量矩阵.
    \end{enumerate}
\end{enumerate}

\clearpage

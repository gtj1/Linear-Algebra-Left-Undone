\chapter{线性空间的运算}

在前述章节中我们对(有限维)线性空间中的基本概念以及研究的基本问题进行了了解. 事实上,很多时候我们还需要研究不同线性空间进行运算后得到的新集合的性质,本节我们将详细展开讨论这一问题.

\section{线性空间的交、并、和}

\begin{definition}{}{}
    设$W_1,W_2$是线性空间$V(\mathbf{F})$的两个子空间,则
    \begin{align*}
        W_1 \cap W_2 & =\{\alpha \mid \alpha\in W_1 \text{~且~} \alpha\in W_2\}            \\
        W_1 \cup W_2 & =\{\alpha \mid \alpha\in W_1 \text{~或~} \alpha\in W_2\}            \\
        W_1 + W_2    & =\{\alpha_1+\alpha_2 \mid \alpha_1\in W_1,\enspace\alpha_2\in W_2\}
    \end{align*}
    分别称为$W_1$和$W_2$的交、并、和.
\end{definition}

交与并的定义实际上与集合交与并的定义类似,而和的定义可能有些许反直觉. 我们可以通过一个例子来体会为什么要定义子空间的和.
\begin{example}{}{子空间运算}
    在$\mathbf{R}^3$中,我们设
    \[\alpha_1=(0,0,1),\ \alpha_2=(0,1,0),\ \alpha_3=(1,0,0).\]
    令$\mathbf{R}^3$子空间$W_1=\spa(\alpha_1,\alpha_2)$,$W_2=\spa(\alpha_1,\alpha_3)$,则$W_1$实际上是$yOz$平面,$W_2$是$xOz$平面,因此我们根据交与并的概念(实际上就是集合取交集和并集)得到$W_1 \cap W_2=\spa(\alpha_1)$(即$z$坐标轴).

    进一步考察并集,事实上显然$W_1 \cup W_2$得到的集合表示$xOz$和$yOz$平面上所有的点. 事实上我们发现,$W_1 \cup W_2$得到的集合关于向量加法、数乘运算并不封闭,例如只需取$\alpha_2+\alpha_3=(0,1,1)$就不在$W_1 \cup W_2$中,因此不再是$\mathbf{R}^3$的子空间.

    接下来我们考察二者之和. 事实上$W_1+W_2=\mathbf{R}^3$. 原因在于
    \begin{enumerate}
        \item $\forall \beta\in W_1 + W_2$,由子空间和的定义可知有$\beta=\beta_1+\beta_2$,其中$\beta_1\in W_1\subseteq \mathbf{R}^3$,$\beta_2\in W_2\subseteq \mathbf{R}^3$,由于$\mathbf{R}^3$是线性空间,其中元素关于加法运算封闭,因此$\beta=\beta_1+\beta_2\in \mathbf{R}^3$,即$W_1+W_2\subseteq \mathbf{R}^3$;

        \item $\mathbf{R}^3$中任一向量$(x,y,z)$总能写成$(x,y,z)=(0,y,z)+(x,0,0)$的形式,其中$(0,y,z)$在$W_1$中,$(x,0,0)$在$W_2$中,因此根据子空间和的定义可知$\mathbf{R}^3\subseteq W_1 + W_2$成立.
    \end{enumerate}
    综上,我们得到$W_1+W_2=\mathbf{R}^3$.
\end{example}

从上面证明$W_1+W_2=\mathbf{R}^3$的过程中我们可以提炼出证明子空间的和等于某一空间的一般方法:本质而言仍然是证明集合相等,因此证明两边包含即可. 证明子空间的和属于某一空间是平凡的,如上述证明的第一部分;第二部分证明某一空间属于子空间和只需要将该空间中任意向量都可分解为各个子空间中向量的和即可.

事实上,根据\autoref{ex:子空间运算} 我们发现,子空间$W_1$和$W_2$的交与和仍然是线性空间,但是它们的并不是线性空间. 事实上,我们可以证明如下定理:
\begin{theorem}{}{}
    设$W_1,W_2$是线性空间$V(\mathbf{F})$的两个子空间,则
    \begin{enumerate}
        \item $W_1 \cap W_2$是$V$的子空间;

        \item $W_1 + W_2$是$V$的子空间;

        \item $W_1 \cup W_2$为$V$的子空间$\iff W_1 \subseteq W_2$或$W_2 \subseteq W_1 \iff W_1 \cup W_2=W_1+W_2$.
    \end{enumerate}
\end{theorem}

\begin{proof}
    我们从子空间的定义出发证明这一定理.
    即验证 $W_1 \cap W_2$ 满足子空间的三个条件:


    \begin{itemize}
        \item
        由于 $W_1, W_2$ 都是 $V$ 的子空间,零向量 $\mathbf{0} \in W_1$ 且 $\mathbf{0} \in W_2$. 因此,$\mathbf{0} \in W_1 \cap W_2$.

        \item
        对于任意的 $x, y \in W_1 \cap W_2$,有 $x \in W_1$ 且 $x \in W_2$,$y \in W_1$ 且 $y \in W_2$.
        由于 $W_1$ 和 $W_2$ 都是 $V$ 的子空间,所以 $x + y \in W_1$ 且 $x + y \in W_2$.
        因此,$x + y \in W_1 \cap W_2$.

        \item
        对于任意的 $x \in W_1 \cap W_2$ 和任意的标量 $\lambda \in \mathbf{F}$,有 $x \in W_1$ 且 $x \in W_2$.
        由于 $W_1$ 和 $W_2$ 都是 $V$ 的子空间,所以 $\lambda x \in W_1$ 且 $\lambda x \in W_2$.
        因此,$\lambda x \in W_1 \cap W_2$.
    \end{itemize}

    所以 $W_1 \cap W_2$ 是 $V$ 的子空间.


    下证$W_1 + W_2$ 是 $V$ 的子空间:
    \begin{itemize}
        \item
        由于 $W_1$ 和 $W_2$ 是 $V$ 的子空间,$\mathbf{0} \in W_1$ 且 $\mathbf{0} \in W_2$.
        因此,$\mathbf{0} = \mathbf{0} + \mathbf{0} \in W_1 + W_2$.

        \item
        对于任意的 $u_1, u_2 \in W_1 + W_2$,存在 $x_1, x_2 \in W_1$ 和 $y_1, y_2 \in W_2$,使得 $u_1 = x_1 + y_1$,$u_2 = x_2 + y_2$.
        则
        $$
        u_1 + u_2 = (x_1 + y_1) + (x_2 + y_2) = (x_1 + x_2) + (y_1 + y_2).
        $$
        由于 $W_1$ 和 $W_2$ 是 $V$ 的子空间,$x_1 + x_2 \in W_1$ 且 $y_1 + y_2 \in W_2$,
        因此,$u_1 + u_2 \in W_1 + W_2$.

        \item
        对于任意的 $u \in W_1 + W_2$ 和标量 $\lambda \in \mathbf{F}$,存在 $x \in W_1$ 和 $y \in W_2$,使得 $u = x + y$.
        则
        $$
        \lambda u = \lambda (x + y) = \lambda x + \lambda y.
        $$
        由于 $W_1$ 和 $W_2$ 是 $V$ 的子空间,$\lambda x \in W_1$ 且 $\lambda y \in W_2$,
        因此,$\lambda u \in W_1 + W_2$.
    \end{itemize}

    因此,$W_1 + W_2$ 也是 $V$ 的子空间.
\end{proof}
我们只在此证明定理的前两条,第三条我们留作习题供读者练习,因为在考试中有出现过. 前两条还可以进行推广,即$V$的有限个子空间的交与和仍然是$V$的子空间.

除此之外,这一定理也告诉我们为什么需要研究子空间的和而更少研究子空间的并:因为子空间的和仍然是线性空间. 直观理解实际上就是和的定义中出现了两个子空间的向量的加法,而构成子空间的核心就是运算封闭,因此这一定义为子空间的和仍构成子空间提供了保证,因此这一定义也是十分自然的.

下面我们来看一个例子,在例子中我们将给出求子空间的和与交的一般方法:
\begin{example}{}{}
    设 $\alpha_1 = (1, 0, -1, 0)$,$\alpha_2 = (0, 1, 2, 1)$,$\alpha_3 = (2, 1, 0, 1)$,是四维实行向量空间 $V$ 中的向量,他们张成的子空间为 $V_1$;又设向量 $\beta_1 = (-1, 1, 1, 1)$,$\beta_2 = (1, -1, -3, -1)$,$\beta_3 = (-1, 1, -1, 1)$ 张成的子空间为 $V_2$,求 $V_1$ 和 $V_2$ 的交与和的基.
\end{example}

\begin{solution}
    \begin{enumerate}
        \item 方法一.  $V_1 +V_2$ 是由 $\alpha_i$ 和 $\beta_i$ 生成的,因此只需要求出这 $6$ 个向量的极大线性无关组即可. 将这 $6$ 个向量按列分块方式拼成矩阵,并用初等行变换将其化为阶梯形矩阵:
              \begin{align*}
                  \begin{pmatrix}
                      1  & 0 & 2 & -1 & 1  & -1 \\
                      0  & 1 & 1 & 1  & -1 & 1  \\
                      -1 & 2 & 0 & 1  & -3 & -1 \\
                      0  & 1 & 1 & 1  & -1 & 1
                  \end{pmatrix}
                  \xrightarrow{}
                  \begin{pmatrix}
                      1 & 0 & 2 & -1 & 1  & -1 \\
                      0 & 1 & 1 & 1  & -1 & 1  \\
                      0 & 2 & 2 & 0  & -2 & -2 \\
                      0 & 0 & 0 & 0  & 0  & 0
                  \end{pmatrix} \\
                  \xrightarrow{}
                  \begin{pmatrix}
                      1 & 0 & 2 & -1 & 1  & -1 \\
                      0 & 1 & 1 & 1  & -1 & 1  \\
                      0 & 0 & 0 & -2 & 0  & -4 \\
                      0 & 0 & 0 & 0  & 0  & 0
                  \end{pmatrix}
              \end{align*}

              所以就可以取 $\alpha_1$,$\alpha_2$,$\beta_1$ 为 $V_1 + V_2$ 的基(不唯一).

              下面再来取 $V_1\cap V_2$ 的基,首先注意到 $\alpha_1$,$\alpha_2$ 是 $V_1$ 的基(从上面的矩阵即可看出),又不难验证 $\beta_1$,$\beta_2$ 是 $V_2$ 的基,$V_2$ 中的向量可以表示为 $\beta_1$,$\beta_2$ 的线性组合. 假设 $t_1\beta_1 + t_2\beta_2$ 属于 $V_1$,则向量组 $\alpha_1, \alpha_2, t_1\beta_1 + t_2\beta_2$ 和向量组 $\alpha_1, \alpha_2$ 的秩相等(因为 $\alpha_1, \alpha_2$ 是 $V_1$ 的基). 因此,我们可以用矩阵方法来求出参数 $t_1, t_2$. 注意到
              \[ \begin{pmatrix}
                      1  & 0 & -t + t_2   \\
                      0  & 1 & t_1 - t_2  \\
                      -1 & 2 & t_1 - 3t_2 \\
                      0  & 1 & -t_1 - t_2
                  \end{pmatrix} \xrightarrow{} \begin{pmatrix}
                      1 & 0 & -t + t_2  \\
                      0 & 1 & t_1 - t_2 \\
                      0 & 2 & -2t_2     \\
                      0 & 0 & 0
                  \end{pmatrix} \xrightarrow{} \begin{pmatrix}
                      1 & 0 & -t + t_2  \\
                      0 & 1 & t_1 - t_2 \\
                      0 & 0 & -2t_1     \\
                      0 & 0 & 0
                  \end{pmatrix} \]

              所以可以得出当且仅当 $t_1 = 0$ 时 $t_1\beta_1 + t_2\beta_2$ 属于 $V_1$,所以 $V_1 \cap V_2$ 的基可取为 $\beta_2$.

        \item 方法二. 求 $V_1 + V_2$ 的基同方法一,现用解线性方程组的方法来求 $V_1 \cap V_2$ 的基. 因为 $\alpha_1$,$\alpha_2$ 是 $V_1$ 的基,$\beta_1$,$\beta_2$ 是 $V_2$ 的基,故对任一向量 $\gamma \in V_1 \cap V_2$,$\gamma = x_1\alpha_1 + x_2\alpha_2 = -x_3\beta_1 - x_4\beta_2$. 因此,求向量 $\gamma$ 等价于求解线性方程组

              \[ x_1\alpha_1 + x_2\alpha_2 + x_3\beta_1 + x_4\beta_2 = 0. \]

              上述线性方程的通解是 $(x_1, x_2, x_3, x_4) = k(-1, 1, 0, 1)$,从而 $\gamma = -k(\alpha_1 - \alpha_2) = -k\beta_2 (k \in \mathbf{R})$,于是 $\beta_2$ 是 $V_1 \cap V_2$ 的基.
    \end{enumerate}
\end{solution}

我们不难发现,两个线性空间的和的求法就是将两个空间的基合并后求极大线性无关组,而交的求法则更具技巧性. 当然这里使用的是简单的向量空间的例子,如果是一般的线性空间,则可以先转化为基下的坐标然后使用上面的方法求解.

当然我们也可以将前面定义的交、并、和的概念推广到有限个子空间的情况,即
\begin{definition}{}{}
    设 $W_1,W_2,\ldots,W_s$ 是线性空间 $V(\mathbf{F})$ 的 $s$ 个子空间,则
    \begin{align*}
        W_1 \cap W_2 \cap \cdots \cap W_s & =\{\alpha \mid \alpha\in W_1 \text{~且~} \alpha\in W_2 \text{~且~} \cdots \text{~且~} \alpha\in W_s\} \\
        W_1 \cup W_2 \cup \cdots \cup W_s & =\{\alpha \mid \alpha\in W_1 \text{~或~} \alpha\in W_2 \text{~或~} \cdots \text{~或~} \alpha\in W_s\} \\
        W_1 + W_2 + \cdots + W_s          & =\{\alpha_1+\alpha_2+\cdots+\alpha_s \mid \alpha_1\in W_1,\enspace\alpha_2\in W_2,\enspace\cdots,\enspace\alpha_s\in W_s\}
    \end{align*}
    分别称为 $W_1,W_2,\ldots,W_s$ 的交、并、和.
\end{definition}

\section{覆盖定理}

本节我们希望探究一个经典问题,即数域上的线性空间是否可以被有限个子空间的并覆盖,下面的覆盖定理就是对这一问题的回答:

\begin{theorem}{覆盖定理}{覆盖定理} \index{fugaidingli@覆盖定理}
    设 $V_1,V_2,\ldots,V_s$ 是数域 $\mathbf{F}$ 上线性空间 $V(\mathbf{F})$ 的 $s$ 个非平凡子空间,证明:$V$ 中至少存在一个向量不属于 $V_1,V_2,\ldots,V_s$ 中的任何一个,即 $V_1 \cup V_2 \cup \cdots \cup V_s\subsetneq V$.
\end{theorem}

覆盖定理表明任何一个数域 $\mathbf{F}$ 上的线性空间都不能被自身有限个非平凡子空间通过并得到. 初看可能有些不够自然,但我们可以从简单的几何意义获得直观的理解:有限条直线的并不可能是一个平面. 下面我们利用数学归纳法进行证明.

\begin{proof}
    \begin{enumerate}
        \item 当$s=2$时,由于$V_1,V_2$是非平凡子空间,因此$V$中存在$\alpha\notin V_1$. 若$\alpha\notin V_2$,则结论已经成立. 若$\alpha\in V_2$,由$V_2$非平凡知存在$\beta\notin V_2$. 我们考虑$\alpha+\beta$和$2\alpha+\beta$,则必有这两个向量都不属于$V_2$(否则有$\beta\in V_2$),并且这两个向量也不能同时属于$V_1$(否则两个向量相减等于$\alpha$也属于$V_1$,矛盾). 这就说明这两个向量中至少有一个既不在$V_1$中也不在$V_2$中,因此结论成立.

        \item 对于$s>2$,假设命题对$s-1$个子空间成立,即$V$中存在向量$\alpha\notin V_1\cup V_2\cup\cdots\cup V_{s-1}$. 若$\alpha\notin V_s$,则结论成立. 若$\alpha\in V_s$,由$V_s$非平凡知存在$\beta\notin V_s$. 我们考虑$\alpha+\beta,2\alpha+\beta,\ldots,s\alpha+\beta$,则与归纳基础中同样的原因,必有这$s$个向量都不属于$V_s$,且这$s$个向量中不可能存在两个向量同属于一个$V_i\enspace(i=1,2,\ldots,s-1)$,因此这$s$个向量中至少有一个不在$V_1\cup V_2\cup\cdots\cup V_s$中,因此结论成立.
    \end{enumerate}
\end{proof}

本质而言$s>2$的情况就是将$s-1$个子空间的并视为一个整体,然后套用$s=2$的情况证明. 若将这一定理的条件限制在$V$为有限维线性空间,我们也可以利用Vandermonde行列式的方法证明,详见\autoref{ex:行列式证明覆盖定理}.我们已经在本章习题A组最后两题为读者准备了覆盖定理的直接证明的题目,下面再给出一个例子供读者应用覆盖定理:
\begin{example}{}{}
    $V_1,V_2,\ldots,V_s$是线性空间$V$的$s$个非平凡子空间,证明:存在$V$的一组基$\alpha_1,\alpha_2,\ldots,\alpha_n$都不在$V_1,V_2,\ldots,V_s$中.
\end{example}

\begin{proof}
    由\nameref{thm:覆盖定理},$V$中存在向量$\alpha_1\notin V_1\cup V_2\cup\cdots\cup V_s$. 继续取$\alpha_2\notin V_1\cup V_2\cup\cdots\cup V_s\cup\spa(\alpha_1)$,则一定有$\alpha_1,\alpha_2$线性无关. 继续取$\alpha_3\notin V_1\cup V_2\cup\cdots\cup V_s\cup\spa(\alpha_1,\alpha_2)$,则一定有$\alpha_1,\alpha_2,\alpha_3$线性无关. 以此类推,最终得到一组基$\alpha_1,\alpha_2,\ldots,\alpha_n$都不在$V_1,V_2,\ldots,V_s$中.
\end{proof}

\section{维数公式}

一个很自然的问题是,两个子空间的和的维数与两个子空间的维数之间是否有关系. 事实上,我们有如下重要的结果:

\begin{theorem}{线性空间维数公式}{线性空间维数公式}
    设$W_1,W_2$是线性空间$V(\mathbf{F})$的两个子空间,则
    \[\dim W_1+\dim W_2=\dim(W_1+W_2)+\dim(W_1\cap W_2).\]
\end{theorem}
上式称为子空间的维数公式,区别于下一专题中的线性映射基本定理的维数公式. 初看这一结论可能会想起中学阶段就已熟知的集合的``容斥原理'',因此记忆上的难度并不大,当然这只是形式上的类似,毕竟这里是研究子空间集合的维数之间的关联,而非集合元素个数的关联. 这一定理的证明思想非常重要,因此我们给出详细的证明.

\begin{proof}
    设$\dim W_1=s,\enspace \dim W_2=t,\enspace \dim(W_1\cap W_2)=r$. 设$W_1\cap W_2$的一组基为$\alpha_1,\alpha_2,\ldots,\alpha_r$,则可以扩充为$W_1$的一组基,记为$\alpha_1,\alpha_2,\ldots,\alpha_r,\beta_1,\ldots,\beta_{s-r}$;也可以扩充为$W_2$的一组基,记为$\alpha_1,\alpha_2,\ldots,\alpha_r,\gamma_1,\ldots,\gamma_{t-r}$. 则我们有
    \[W_1+W_2=\spa(\alpha_1,\ldots,\alpha_r,\beta_1,\ldots,\beta_{s-r},\gamma_1,\ldots,\gamma_{t-r})\]
    (如果对这一步有疑问可以回顾\autoref{ex:子空间运算} 中的证明). 由此,我们要证$\dim (W_1+W_2)=s+t-r$,只需证$\alpha_1,\ldots,\alpha_r,\beta_1,\ldots,\beta_{s-r},\gamma_1,\ldots,\gamma_{t-r}$线性无关. 为此,我们设
    \begin{equation}\label{eq:4:维数公式证明1}
        a_1\alpha_1+\cdots+a_r\alpha_r+b_1\beta_1+\cdots+b_{s-r}\beta_{s-r}+c_1\gamma_1+\cdots+c_{t-r}\gamma_{t-r}=0,
    \end{equation}
    即
    \begin{equation}\label{eq:4:维数公式证明2}
        a_1\alpha_1+\cdots+a_r\alpha_r+b_1\beta_1+\cdots+b_{s-r}\beta_{s-r}=-c_1\gamma_1-\cdots-c_{t-r}\gamma_{t-r}.
    \end{equation}
    显然,\autoref{eq:4:维数公式证明2} 等号两端的向量分别属于$W_1$和$W_2$,因此它们都属于$W_1\cap W_2$,因此都可以被$W_1\cap W_2$的基线性表示,即
    \[-c_1\gamma_1-\cdots-c_{t-r}\gamma_{t-r}=d_1\alpha_1+\cdots+d_r\alpha_r,\]
    即
    \begin{equation}\label{eq:4:维数公式证明3}
        c_1\gamma_1+\cdots+c_{t-r}\gamma_{t-r}+d_1\alpha_1+\cdots+d_r\alpha_r=0.
    \end{equation}
    由于$\alpha_1,\ldots,\alpha_r,\gamma_1,\ldots,\gamma_{t-r}$是$W_2$的基,因此\autoref{eq:4:维数公式证明3} 所有系数都为0,即$c_1=\cdots=c_{t-r}=d_1=\cdots=d_r=0$. 代入\autoref{eq:4:维数公式证明2} 后,由于$\alpha_1,\ldots,\alpha_r,\beta_1,\ldots,\beta_{s-r}$是$W_1$的基,因此可得$a_1=\cdots=a_r=b_1=\cdots=b_{s-r}=0$,因此,代入\autoref{eq:4:维数公式证明1} 后可知$\alpha_1,\ldots,\alpha_r,\beta_1,\ldots,\beta_{s-r},\gamma_1,\ldots,\gamma_{t-r}$必定线性无关(因为根据前述证明所有系数只能为0),故得证.
\end{proof}

总结而言,这一定理证明用到的思想就是``设小扩大''. 我们设出最小空间$V_1\cap V_2$的基,然后分别扩充为$V_1$和$V_2$的基,然后观察要证明的等式和已知的联系,然后利用\autoref{eq:4:维数公式证明2} 构造等式两边属于不同空间的向量这一技巧即可. 这一``设小扩大''的技巧我们之后还会很多见,例如\autoref{thm:线性映射基本定理}以及\autoref{ex:维数公式技巧例题}.

\section{线性空间的直和}

我们将来证明或者利用和空间时,很多时候都是利用和空间定义进行向量分解. 我们特别重视分解唯一时的情形,因为这对我们的研究很有帮助,这时的和即为直和. 严谨而言,我们有如下定义:
\begin{definition}{}{}
    设$W_1,W_2$是线性空间$V(\mathbf{F})$的两个子空间. 若$W_1 \cap W_2=\{0\}$,则$W_1+W_2$叫做$W_1$与$W_2$的\term{直和}\index{zhihe@直和 (direct sum)},记作$W_1\oplus W_2$.

    进一步地,若$V=W_1\oplus W_2$,则称$W_1,W_2$为\term{互补子空间}\index{xianxingkongjian!hubu@互补子空间 (complementary subspaces)},或$W_1$是$W_2$的补空间,或$W_2$是$W_1$的补空间.
\end{definition}

直和有以下等价的命题,我们证明或者利用直和都可以任意选择:
\begin{theorem}{}{直和等价命题}
    对于子空间$W_1,W_2$,下列命题等价:
    \begin{enumerate}
        \item $W_1+W_2$是直和,即$W_1 \cap W_2=\{\vec{0}\}$;

        \item $W_1+W_2$中的每个向量$\alpha$的分解式$\alpha=\alpha_1+\alpha_2\enspace(\alpha_1\in W_1,\enspace\alpha_2\in W_2)$唯一;

        \item 零向量的分解式$\vec{0}=\alpha_1+\alpha_2 \enspace(\alpha_1\in W_1,\enspace\alpha_2\in W_2)$仅当$\alpha_1=\alpha_2=\vec{0}$时成立;

        \item $\dim (W_1+W_2)=\dim W_1+\dim W_2$.
    \end{enumerate}
\end{theorem}

定理的证明是基本的.
\begin{proof}
    上述命题等价,只需证明 (1) $\implies$ (2) $\implies$ (3) $\implies$ (4) $\implies$ (1).
    \begin{enumerate}
        \item 先证 (1) $\implies$ (2): 设 $W_1 + W_2$ 中的 $\alpha$ 有两个分解式
        $$
        \alpha = \alpha_1 + \alpha_2 = \beta_1 + \beta_2, \quad \alpha_1, \beta_1 \in W_1, \ \alpha_2, \beta_2 \in W_2,
        $$
        则 $\alpha_1 - \beta_1 = \beta_2 - \alpha_2 \in W_1 \cap W_2 = \{\mathbf{0}\}$,于是得 $\beta_1 = \alpha_1, \beta_2 = \alpha_2$,故 $\alpha$ 的分解式是唯一的.
        \item 其次证 (2) $\implies$ (3): 由 $W_1 + W_2$ 中零向量的分解式唯一性以及 $\vec{0} = \vec{0} + \vec{0}$,立即得命题 (3) 成立.
        \item 再证 (3) $\implies$ (4): 由命题 (3) 可推出 $W_1 \cap W_2 = \{\mathbf{0}\}$,因为若 $W_1 \cap W_2 \ne \{\mathbf{0}\}$,则存在 $\vec{0} \ne a \in W_1 \cap W_2$,使得 $\vec{0} = \alpha + (-\alpha)$,(其中 $\alpha \in W_1, -\alpha \in W_2$),这与命题 (3) 相矛盾. 再根据维数公式 就得命题 (4).
        \item 最后由命题 (4) 及维数公式 立即得命题 (1) 成立.
    \end{enumerate}
\end{proof}
在实际运用中我们要非常熟悉这些等价条件,因为都可能使用到.

我们也可以定义有限个子空间的直和,即$V=W_1\oplus W_2\oplus\cdots\oplus W_n \iff W_i \cap \sum\limits_{j \neq i}W_j=\{\vec{0}\}$,即每个子空间与其余子空间的和的交都是$\{\vec{0}\}$. 等价命题也是上述定理的推广,例如唯一分解、$\vec{0}$的分解以及维数公式推广,此处不再赘述. 除此之外,我们还有一个与多空间直和相关的定理:
\begin{theorem}{}{多空间直和}
    若$V=V_1\oplus V_2,\enspace V_1=V_{11}\oplus\cdots\oplus V_{1s},\enspace V_2=V_{21}\oplus\cdots\oplus V_{2t}$,则
    \[V=V_{11}\oplus\cdots\oplus V_{1s}\oplus V_{21}\oplus\cdots\oplus V_{2t}\]
\end{theorem}
这一定理的证明是很简单的,实际上利用零向量分解唯一即可,这里不再赘述. 进一步地,我们可以定义无穷个子空间的直和,推广\autoref{thm:直和等价命题}的 $2$,无穷个子空间的直和结果中的元素也可以表达为每个子空间的元素之和,但我们需要额外要求和式中只有有限项非零. 直观地,这其实是为了保证和式的收敛性.

在习题中我们证明直和一般有两种思路,一种是先证和,再证直和,我们来看一个例子:
\begin{example}{}{}
    考虑 $\mathbf{R}$上(即定义在实数域上)的所有实值函数(即取值于实数域)构成的集合 $\mathbf{R}^\mathbf{R} = \{f \mid f: \mathbf{R} \to \mathbf{R}\}$.  函数 $f: \mathbf{R} \to \mathbf{R}$ 被称为偶函数,若 $f(-x) = f(x)$ 对所有 $x \in \mathbf{R}$ 成立;函数 $f: \mathbf{R} \to \mathbf{R}$ 被称为奇函数,若 $f(-x) = -f(x)$ 对所有 $x \in \mathbf{R}$ 成立. 令 $V_e$ 代表 $\mathbf{R}$ 上的实值(函数取值于实数域)偶函数构成的集合,$V_o$ 代表 $\mathbf{R}$ 上的实值奇函数构成的集合,证明:$\mathbf{R}^\mathbf{R} = V_e \oplus V_o$.
\end{example}

\begin{proof}
    \begin{enumerate}
        \item 首先我们证明 $\mathbf{R}^\mathbf{R} = V_e + V_o$,因为 $V_e + V_o \subseteq \mathbf{R}^\mathbf{R}$ 是显然的,所以只需证明 $\mathbf{R}^\mathbf{R} \subseteq V_e + V_o$ 即可,即证明任意实值函数 $f: \mathbf{R} \to \mathbf{R}$ 都可以表示为一个偶函数与一个奇函数之和. 令 $f: \mathbf{R} \to \mathbf{R}$ 为任意实值函数,定义
        \[ f_e(x) = \frac{f(x) + f(-x)}{2}, \quad f_o(x) = \frac{f(x) - f(-x)}{2}, \]
        根据定义不难验证 $f_e(x)$ 为偶函数,$f_o(x)$ 为奇函数,且 $f(x) = f_e(x) + f_o(x)$ 对任意 $x \in \mathbf{R}$ 成立,故任意实值函数 $f$ 都可以表示为一个偶函数与一个奇函数之和,故 $\mathbf{R}^\mathbf{R} = V_e + V_o$ 成立.
        \item 再证明和为直和,我们有很多种证明方式,对应于直和等价条件的前三条,最后一条因为 $\mathbf{R}^\mathbf{R}$ 显然是无限维空间,因此不能使用.

        我们考虑证明交集为 $\{\vec{0}\}$,因为这比较容易. 设 $f(x) \in V_e \cap V_o$,则 $f(x)$ 既是偶函数又是奇函数,即 $f(x) = f(-x) = -f(x)$,解得 $f(x) = 0$ 对任意的 $x \in \mathbf{R}$ 成立,故 $f(x)$ 为零函数,这显然是 $\mathbf{R}^\mathbf{R}$ 中的零元,故 $V_e \cap V_o = \{\vec{0}\}$,故我们有 $V_e \oplus V_o = \mathbf{R}^\mathbf{R}$.
    \end{enumerate}
\end{proof}

我们也可以首先证明分解唯一,然后证明直和,但相对而言更加复杂. 反之,由于我们已经利用交集为 $\{\vec{0}\}$ 得到了 $V_e \oplus V_o = \mathbf{R}^\mathbf{R}$,现在我们可以``曲线救国''式地证明任意实值函数 $f$ 都可以唯一表示为一个偶函数与一个奇函数之和.

还有一种证明 $V = V_1 \oplus V_2$ 的方式是先令 $W = V_1 + V_2$,先用直和等价条件证明和为直和,再证 $W = V$ 即可,下面是一个例子:
\begin{example}{}{}
    设 $V_1, V_2$ 分别是 $\mathbf{R}$ 上齐次线性方程组 $x_1 = x_2 = \cdots = x_n = 0$ 和 $x_1 + x_2 + \cdots + x_n = 0$ 的解空间,证明 $\mathbf{R}^n = V_1 \oplus V_2$.
\end{example}

\begin{proof}
    设 $W = V_1 + V_2$. 对于任意向量 $x = (x_1,\cdots,x_n) \in V_1 \cap V_2$,必然同时满足 $V_1, V_2$ 对应的两个方程组,即是下述方程组的解:
    \[\begin{cases}
            x_1 + \cdots + x_n = 0 \\
            x_1= \cdots = x_n
        \end{cases} \implies x_1 = \cdots = x_n = 0,\]
    故 $V_1 \cap V_2 = \{\vec{0}\}$,由直和条件可知,$W = V_1 \oplus V_2$. 下证$W = \mathbf{F}^n$,因为 $W \subset \mathbf{F}^n$,只需证明维数相同即可,根据\autoref{ex:线性方程组的解的维数},$V_1$ 的维数为 $n - 1$,$V_2$ 的维数为 $n - 1$,由于 $W = V_1 \oplus V_2$,有 $\dim W = \dim V_1 + \dim V_2 = n$,故 $W = \mathbf{F}^n$ 得证.
\end{proof}

总结这一解法的思路,令 $W = V_1 + V_2$,先证明 $W = V_1 \oplus V_2$. 而我们知道 $W$ 必然是全空间 $V$ 的一个子空间,因此只需证明 $\dim W = \dim V$ 即可证明 $W = V$,即得 $V = V_1 \oplus V_2$.

最后我们要提醒读者注意的是,有限维线性空间的一个子空间的补空间并不唯一,如下面的例子:
\begin{example}{}{}
    在$\mathbf{R}^3$中,$W_1=\spa(\alpha_1)$,则其补空间根据直和的维数公式可知为2,记为$W_2=\spa(\alpha_2,\alpha_3)$. 实际上只需要$\alpha_1,\alpha_2,\alpha_3$线性无关即可,事实上这样的选择是有无穷种的,因为$W_1$本质表示一条直线,故$W_2$是不包含$W_1$且不与$W_1$平行的平面即可,这样$\alpha_2,\alpha_3$是$W_2$任意一组基都可以.
\end{example}

\section{商空间}

这一节我们需要引入代数中一个非常重要的运算,即某个代数结构的商. 在第一讲中我们给出了一个非常重要的概念——等价类,这里的目标是在一个线性空间 $V$ 中定义出等价关系,即利用\autoref{thm:等价类的性质},以某种方式将一个线性空间中的所有向量划分为几个不相交的等价类的并集,最后在此基础上定义出线性空间的商运算.

\subsection{从等价关系出发}

为了定义出这个等价关系,我们首先需要确定的是它应当满足怎样的性质,否则对于如何定义这一等价关系我们将毫无头绪. 这一问题的出发点事实上就隐藏在\autoref{ex:有限域} 后关于``相容''的讨论中. 在那里,我们要求从一般整数加法和乘法继承下来的模$n$剩余类上的加法和乘法是良定义的,类比于这里的目标,则是希望线性空间中等价类的集合上(也就是商集)定义的加法和数乘运算,可以自然地继承一般的向量加法和乘法,并且保证相容性.

下面我们开始形式化地把上面抽象的描述转化为表达式. 我们需要在线性空间$V$中定义一个等价关系$R$,得到一个等价类构成的商集:
\[V/R=\{\overline{v_1},\overline{v_2},\ldots,\overline{v_n},\ldots\},\]
其中$\overline{v_i}$表示$v_i\in V$所在的等价类,取$v_i$为代表元. 需要注意的是,这里的等价类不一定是有限个,因此最后还是省略号.

接下来我们需要定义等价类之间的运算,我们希望自然继承线性空间的加法和数乘运算,故我们定义商集上的加法和数乘运算满足对于任意的$\overline{v_1},\overline{v_2}\in V/R$和$\lambda\in\mathbf{F}$,有
\begin{equation} \label{eq:10:商集运算}
    \begin{gathered}
        \overline{v_1}+\overline{v_2}=\overline{v_1+v_2},\\
        \lambda\overline{v_1}=\overline{\lambda v_1}.
    \end{gathered}
\end{equation}
需要注意的是,等号左边的运算是商集$V/R$上的,右边是线性空间$V$上的,这里不再像模$n$剩余类上的运算定义那样使用不同的符号(例如加法改成$\oplus$)是出于习惯,以及相信读者学习到今天应当适应了为记号上方便做出的牺牲(比如对于任何线性空间,即使加法不是定义成最一般的加法,但也写成加号的形式).

最后我们需要上面的运算保证相容性,也就是说,对于任意的$v_1,v_2,u_1,u_2\in V$,如果$v_1Rv_2$,$u_1Ru_2$则$(v_1+u_1)R(v_2+u_2)$,$\lambda v_1R\lambda v_2$,展开写为:
\begin{gather*}
    \overline{v_1+u_1}=\overline{v_2+u_2},\\
    \overline{\lambda v_1}=\overline{\lambda v_2}.
\end{gather*}

推进到这里或许我们还是很难看出如何定义这一等价关系,但我们可以从简单的角度入手,逐步观察这些等价类的特点. 我们可以首先考虑线性空间的零向量所在的等价类具有什么性质. 事实上,我们很容易得到如下定理:
\begin{theorem}{}{}
    线性空间$V$的零向量所在的等价类$\overline{0}$一定是$V$的子空间.
\end{theorem}
\begin{proof}
    \begin{enumerate}
        \item 加法封闭:$\forall \alpha,\beta\in\overline{0}$,则$\alpha\,R\,0$,$\beta\,R\,0$,因此根据相容性,$\alpha+\beta\,R\,0$,即$\alpha+\beta\in\overline{0}$;
        \item 数乘封闭:$\forall \alpha\in\overline{0}$,则$\alpha\,R\,0$,$\lambda\in\mathbf{F}$,因此根据相容性,$\lambda\alpha\,R\,0$,即$\lambda\alpha\in\overline{0}$.
    \end{enumerate}
\end{proof}

这一结论非常关键,它使得我们把抽象的等价关系与一个子空间绑定. 我们记这一子空间为$U$,即$\overline{0}=U$,于是下面这一结论也是容易得到的:
\begin{theorem}{}{}
    设$v_1,v_2\in V$,则$v_1Rv_2$当且仅当$v_1-v_2\in U$.
\end{theorem}
\begin{proof}
    \begin{enumerate}
        \item ($\implies$) 直接取$u_1,u_2\in U$,即$u_1\,R\,0$,$u_2\,R\,0$,因此根据相容性,$\overline{v_1}+\overline{u_1}=\overline{v_2}+\overline{u_2}$,移项得$\overline{v_1}-\overline{v_2}=\overline{u_2}-\overline{u_1}$,注意到$\overline{v_1}-\overline{v_2}=\overline{v_1}+(-1)\cdot\overline{v_2}=\overline{v_1}+\overline{-v_2}=\overline{v_1-v_2}$,同理有$\overline{u_2}-\overline{u_1}=\overline{u_2-u_1}$,因此$\overline{v_1-v_2}=\overline{u_2-u_1}=\overline{0}$,即$v_1-v_2\in U$;
        \item 设$v_1-v_2=u\in U$,则$\overline{v_1}=\overline{v_2}+\overline{u}=\overline{v_2}+\overline{0}=\overline{v_2}$,因此$v_1Rv_2$.
    \end{enumerate}
\end{proof}

至此,我们完成了对线性空间$V$上的符合相容性的等价关系$R$的性质的讨论. 我们发现,尽管运算定义和相容性的要求非常抽象,但是在线性空间的背景下,我们成功地将$R$与一个子空间$U$对应起来,这个子空间实际上就是等价类$\overline{0}$. 反过来,当我们需要定义线性空间的等价类的时候,我们可以从一个子空间$U$出发,然后定义等价关系$R$为
\begin{equation} \label{eq:10:线性空间等价关系}
    \forall\alpha,\beta\in V,\enspace\alpha\,R\,\beta\iff \alpha-\beta\in U.
\end{equation}
事实上,验证这一关系的确是等价关系是非常简单的:
\begin{enumerate}
    \item (自反性) $\forall \alpha\in V,\enspace\alpha-\alpha=0\in U$,故$\alpha\,R\,\alpha$;

    \item (对称性) $\forall \alpha,\beta\in V,\enspace\alpha\,R\,\beta\implies \alpha-\beta\in U\implies \beta-\alpha=-(\alpha-\beta)\in U\implies \beta\,R\,\alpha$;

    \item (传递性) $\forall \alpha,\beta,\gamma\in V,\enspace\alpha\,R\,\beta,\enspace\beta\,R\,\gamma\implies \alpha-\beta\in U,\enspace\beta-\gamma\in U\implies \alpha-\gamma=(\alpha-\beta)+(\beta-\gamma)\in U\implies \alpha\,R\,\gamma$.
\end{enumerate}
基于此,下一节开始我们将正式给出商空间的定义. 我们可以首先可以定义商集$V/R$是由等价类构成的集合,在线性空间的背景下,由于我们知道$R$是从一个子空间$U$出发的,因此我们也将商集记为$V/U$. 我们将按照自然的方式定义商集中的元素的加法和数乘运算,并证明实际上商集可以构成线性空间,于是称其为商空间,下面我们开始我们严格的陈述.

\subsection{仿射子集与商空间}

紧接着上一节末尾的思路,我们首先定义线性空间等价关系的商集. 研究商集,事实上首先需要研究等价类的性质. 回顾\autoref{eq:10:线性空间等价关系},我们知道向量$\alpha\in V$所在的等价类为:
\[\overline{\alpha}=\{\beta\in V \mid \beta\,R\,\alpha\}=\{\beta\in V \mid \beta-\alpha\in U\}=\{\beta\in V \mid \beta=\alpha+\gamma,\enspace\gamma\in U\}\]
最后一个集合还可以进一步写成$\{\alpha+\gamma \mid \gamma\in U\}$,我们记为$\alpha+U$,称之为$V$的仿射子集. 我们给出如下完整的定义:
\begin{definition}{仿射子集}{} \index{fangsheziji@仿射子集 (affine subset)}
    设$v\in V$,$U$是$V$的子空间,则$V$的\term{仿射子集}是$V$的形如$v+U$的子集,其中$v+U$定义为
    \[v+U=\{v+u \mid u\in U\}.\]
\end{definition}
我们知道,仿射子集就是我们在线性空间上定义的等价关系的等价类. 基于等价类的性质,我们有如下定理:
\begin{theorem}{}{}
    设$U$是$V$的子空间,$v,w\in V$,则以下陈述等价:
    \begin{enumerate}
        \item $v-w\in U$;
        \item $v+U=w+U$;
        \item $(v+U)\cap(w+U)\neq \varnothing$.
    \end{enumerate}
\end{theorem}

还需要强调的一点是,$(v+U)+(w+U)$与$(v+w)+U$是完全相同的集合,等价性是显然的,我们只需要展开写出仿射子集定义然后证明两个集合互相包含即可. 当然更一般的情形为
\[(v_1+U)+(v_2+U)+\cdots+(v_n+U)=(v_1+v_2+\cdots+v_n)+U.\]
相信读者对``仿射''一词并不完全陌生,仿射变换实际上就是形如\[\vec{y}=A\vec{x}+\vec{b}\]的映射,其中$\vec{y},\vec{x},\vec{b}$为向量,$A$是一个矩阵. 实际上一元向量的情况就对应着一条斜率为$A$截距为$b$的直线. 事实上,若$V$为二维空间(平面),$U$为$V$的一维子空间,则其几何意义就是一条过原点的直线,而集合$v+U$实际上将原集合所有点沿着$v$的方向平移,可以得到截距不为0的直线,这就体现了``仿射''一词的意义. 高维空间则是同理,只是我们很难直观地看到这一点. 因此,我们也可以称仿射子集$v+U$\textbf{\heiti 平行于}$U$. 当然,在我们讨论完对偶后,我们会再来审视仿射子集更深的含义.

下面的例子给出了仿射子集的一种等价描述,基于此我们可以对仿射子集中向量的结构有更进一步的了解:
\begin{example}{}{仿射子集性质}
    证明:$V$的非空子集$A$是$V$的仿射子集当且仅当对所有的$v,w\in A$和$\lambda\in\mathbf{F}$均有$\lambda v+(1-\lambda)w\in A$.
\end{example}

\begin{proof}
    \begin{enumerate}
        \item 若已知 $A$ 是 $V$ 的仿射子集,则 $A=\alpha+U$,其中 $\alpha\in V$ 且 $U$ 是 $V$ 的子空间.因此,对任意 $\lambda \in \mathbf{F}$,
        和任意的 $v=\alpha+u_1,w=\alpha+u_2 \in A$,有
        \[
            \lambda v+(1-\lambda)w=\lambda(\alpha+u_1)+(1-\lambda)(\alpha+u_2)=\alpha+\lambda u_1+(1-\lambda)u_2 \in A
        \]
        因此对于仿射子集中的元素$v,w$具有上述性质.
        \item 若已知$\lambda v+(1-\lambda)w \in A$,令$\alpha \in A$,则
        \[
            A-\alpha=\{v-\alpha|v\in A\}
        \]
        我们的目标是证明这是一个子空间;此时我们有$\lambda x + (1 - \lambda) \alpha \in A$,这是因为我们的假设是$\lambda v + (1 - \lambda) w \in A$;
        即$\lambda (x-\alpha) \in A- \alpha $,这样就证明了数乘封闭;对于$x-\alpha,y-\alpha \in A-\alpha$,

        \[
           x-\alpha+y-\alpha=(x+y)-2\alpha=2(\dfrac{x+y}{2}-\alpha) \in A-\alpha
        \]

        这里运用了数乘封闭来证明加法封闭性;因此$A-\alpha$是一个子空间.
    \end{enumerate}
\end{proof}

事实上,结合我们之前所说的仿射子集几何意义,这一结论在平面上来看正是我们高中学习的平面向量中学习的三点共线的等价条件的同义表达:
\begin{theorem}{}{}
    设$P,A,B,C$是平面上四点,$P$与$A,B$不共线,则$C$与$A,B$共线等价于存在$\lambda\in\mathbf{R}$使得$\overrightarrow{PC}=\lambda\overrightarrow{PA}+(1-\lambda)\overrightarrow{PB}$.
\end{theorem}
同时我们发现仿射子集实际上是我们在数学分析或微积分学习的凸集的特殊形式,在凸集中我们只要求$\lambda\in[0,1]$,这里我们要求整个数域上的点都要有\autoref{ex:仿射子集性质} 所述的性质. 当然这不是线性代数中研究的内容,感兴趣的同学可以学习凸优化的相关课程进一步了解.

事实上在习题中我们将给出\autoref*{ex:仿射子集性质} 更一般的形式,我们可以回忆\autoref{thm:线性扩张构造子空间},就会发现仿射子集的结构和线性空间保留了一些相似性,即虽然不能像线性空间一样保证加法数乘运算封闭,但仿射子集一定是保证凸组合封闭的集合.

定义了等价类(即仿射子集)并研究了其性质后,我们可以定义相应的商集(即由全体等价类构成的集合),我们称之为商空间:
\begin{definition}{}{}
    设$U$是$V$的子空间,则商空间$V/U$是指所有由\autoref{eq:10:线性空间等价关系} 诱导的等价类构成的集合,即$V$的所有平行于$U$的仿射子集的集合,即
    \[V/U=\{v+U \mid v\in V\}.\]
\end{definition}
我们希望这些这一商集(商空间)真的构成线性空间,因此还需要定义加法和数乘运算. 定义则与上一节中的要求一样,是自然继承向量加法而来的,即满足\autoref{eq:10:商集运算},我们把式中等价类写为仿射子集的形式即可得到如下定义:
\begin{definition}{}{}
    设$U$是$V$的子空间,则商空间$V/U$上的加法和数乘运算定义为:$\forall \alpha,\beta\in V$和$\lambda\in\mathbf{F}$,
    \begin{gather*}
        (\alpha+U)+(\beta+U)=(\alpha+\beta)+U, \\
        \lambda(\alpha+U)=(\lambda\alpha)+U.
    \end{gather*}
\end{definition}
我们很容易根据线性空间8条性质验证商空间在上述加法和数乘运算定义下构成线性空间,在此不再赘述. 特别注意这一线性空间的零向量是特别的,应当为$U$(即$\vec{0}+U$,一定注意不是$\vec{0}$,读者在验证商空间是线性空间时就会发现).

正常而言,在定义了一个线性空间后我们自然地想了解它的基本结构——基和维数,商空间也不例外. 我们有很多的角度来得到相关的结论,这里首先介绍一个直接的方式得到关于维数的结论,其余的方法我们将在介绍完线性映射和对偶空间后讨论.

\begin{theorem}{商空间的维数公式}{商空间的维数公式}
    设$U$是有限维线性空间$V$的子空间,则
    \[\dim V/U=\dim V-\dim U.\]
\end{theorem}

这一定理的形式与维数公式完全类似,都是几个线性空间之间的维数的等式关系,因此我们有理由相信证明思想也会是类似的,即``设小扩大'':

\begin{proof}
    取$U$的一组基$\alpha_1,\alpha_2,\ldots,\alpha_s$,将其扩充为$V$的一组基$\alpha_1,\alpha_2,\ldots,\alpha_s,\alpha_{s+1},\ldots,\alpha_n$. 于是我们要证的转化为$\dim V/U=n-s$,即证明$V/U$的一组基的长度为$n-s$.

    类似于线性映射基本定理的证明,我们可以依靠直觉猜想. 我们猜想$V/U$的一组基为$\{\alpha_{s+1}+U,\alpha_{s+2}+U,\ldots,\alpha_n+U\}$. 这是很自然的想法. 我们只需要验证这组基的两个条件:线性无关和张成性:
    \begin{enumerate}
        \item 线性无关:设$\lambda_{s+1},\lambda_{s+2},\ldots,\lambda_n\in\mathbf{F}$,使得
              \[\lambda_{s+1}(\alpha_{s+1}+U)+\lambda_{s+2}(\alpha_{s+2}+U)+\cdots+\lambda_n(\alpha_n+U)=U.\]
              特别注意这里的零元是$\vec{0}+U=U$,实际上,上式等价于
              \[(\lambda_{s+1}\alpha_{s+1}+\lambda_{s+2}\alpha_{s+2}+\cdots+\lambda_n\alpha_n)+U=U.\]
              根据仿射子集定义,$\lambda_{s+1}\alpha_{s+1}+\lambda_{s+2}\alpha_{s+2}+\cdots+\lambda_n\alpha_n\in U$,因此可以被表示为$U$的基的线性组合,即
              \[\lambda_{s+1}\alpha_{s+1}+\lambda_{s+2}\alpha_{s+2}+\cdots+\lambda_n\alpha_n=\mu_1\alpha_1+\mu_2\alpha_2+\cdots+\mu_s\alpha_s.\]
              于是我们有
              \[\lambda_{s+1}\alpha_{s+1}+\lambda_{s+2}\alpha_{s+2}+\cdots+\lambda_n\alpha_n-\mu_1\alpha_1-\mu_2\alpha_2-\cdots-\mu_s\alpha_s=0.\]
              由于$\alpha_1,\alpha_2,\ldots,\alpha_n$是$V$的一组基,因此我们有$\lambda_{s+1}=\lambda_{s+2}=\cdots=\lambda_n=\mu_1=\mu_2=\cdots=\mu_s=0$. 从而$\alpha_{s+1}+U,\alpha_{s+2}+U,\ldots,\alpha_n+U$线性无关;

        \item 张成空间:$\forall\alpha+U\in V/U$,其中$\alpha\in V$,我们有$\alpha$可以被$V$的基线性表示为
              \[\alpha=\lambda_1\alpha_1+\lambda_2\alpha_2+\cdots+\lambda_n\alpha_n.\]
              于是
              \begin{align*}
                  \alpha+U & =(\lambda_1\alpha_1+\lambda_2\alpha_2+\cdots+\lambda_n\alpha_n)+U         \\
                           & =(\lambda_1\alpha_1+U)+(\lambda_2\alpha_2+U)+\cdots+(\lambda_n\alpha_n+U) \\
                           & =\lambda_1(\alpha_1+U)+\lambda_2(\alpha_2+U)+\cdots+\lambda_n(\alpha_n+U)
              \end{align*}
              因此$V/U$中任意元素均可被$\alpha_{s+1}+U,\alpha_{s+2}+U,\ldots,\alpha_n+U$线性表示,即$\alpha_{s+1}+U,\alpha_{s+2}+U,\ldots,\alpha_n+U$张成$V/U$.
    \end{enumerate}
\end{proof}

由此我们知道了商空间的维数表达式,也在通过证明过程知道了如何得到商空间的一组基. 下面我们来计算一个例子:

\begin{example}{}{}
    设$A$是$\mathbf{R}$上的$2\times 3$矩阵:
    \[A=\begin{pmatrix}
            1 & -1 & 2 \\ 1 & 0 & -1
        \end{pmatrix}.\]
    \begin{enumerate}
        \item 求齐次线性方程组$AX=0$的解空间$W$的一组基;

        \item 求商空间$\mathbf{R}^3/W$的维数和一组基.
    \end{enumerate}
\end{example}

\begin{solution}
    \begin{enumerate}
        \item 首先求 $AX=0$ 的解,得到其基础解系为$k(1,3,1)^{\mathrm{T}},k \in \mathbf{R}$,故其解空间的一组基为$\{(1,3,1)^{\mathrm{T}}\}$.
        \item 根据\autoref{thm:商空间的维数公式}的证明过程,我们将 $W$ 的基扩张为 $\mathbf{R^3}$ 的一组基,如 $\{(1,3,1)^{\mathrm{T}},(1,0,0)^{\mathrm{T}},(0,1,0)^{\mathrm{T}}\}$,然后将扩张的基中``加上''$W$ 变成仿射子集,即可得到商空间的一组基:
        \[
            \{(1,0,0)^{\mathrm{T}}+W,(0,1,0)^{\mathrm{T}}+W\}
        \]
        维数为 $2$. 这可以从基的数量得到,也可以由 $\mathbf{R}^3$ 的维数减去 $W$ 的维数得到.

    \end{enumerate}
\end{solution}

\begin{summary}

    本讲我们首先介绍了线性空间之间的三种运算——交、并、和. 和的概念初次见到可能有些许抽象,但经过一些例子之后我们应当能理解为什么线性空间不同于普通集合,更常用``和''这一运算. 关于并我们给出了一些构成线性空间的条件以及一个重要的覆盖定理,读者了解即可. 关于交与和我们给出了一个维数公式,它不仅结论非常重要,``设小扩大''的证明思想也是在未来非常常用的. 进一步地,我们讨论了直和的概念以及它的等价条件,以及证明直和的两种思路. 我们必须要重视直和这一概念,因为它在未来关于线性变换矩阵约化表示的讨论中起到重要的桥梁作用.

    商运算是代数学中的一种重要的运算,我们从``如何在线性空间中定义相容的等价类上的运算''出发,得到了线性空间上等价关系的定义方式,从而进一步定义了仿射子集和商空间的概念:我相信只要读者理解了等价类的相关内容,商空间也应当是不会太抽象的. 然后我们讨论了商空间的维数公式,这与线性空间交与和的维数公式证明有异曲同工之妙. 当然这一公式的证明有非常多的方式,我们将在线性映射、对偶空间的讨论中再次回顾这一公式.

    除此之外,线性空间之间还有一种基于笛卡尔积的运算,我们将在后续讨论同构的时候作为一个应用讨论,将直和与笛卡尔积联系起来.

\end{summary}

\begin{exercise}
    \exquote[H. 庞加莱(Henri Poincaré)]{在选择了一套恰当的语言之后,我们会惊讶地发现,所有对于已知对象的阐述都能被立刻推广到许多新的对象上:不需要进行任何改写,甚至包括术语,因为在这种语言下,所有的名字也是一致的。}

    \begin{exgroup}
        \item 设$V=\{(a_1,a_2,a_3,a_4) \mid a_1+a_2+a_3+a_4=0\}$,$W=\{(a_1,a_2,a_3,a_4) \mid a_1-a_2-a_3+a_4=0,a_1+a_2+a_3-a_4=0\}$.
        \begin{enumerate}
            \item 证明:$V$和$W$为$\mathbf{R}^4$的子空间;
            \item 分别求$V \cap W$,$V+W$以及$W$的补空间的维数与一组基.
        \end{enumerate}
        \begin{answer}
            \begin{enumerate}
                \item 显然任意$\alpha\in v,\beta \in w$, 有 $\alpha,\beta\in \mathbf{R}^4$ 成立,因此只需证明 $v,w$ 封闭即可. 由于 $v,w$ 内向量的约束条件都是齐次线性方程,封闭性自然满足,则 $v,w$ 都是 $\mathbf{R}^4$ 的子空间.

                \item \begin{enumerate}
                          \item $V\cap W=\{(a_1,a_2,a_3,a_4) \mid a_1+a_2+a_3+a_4=0,a_1-a_2-a_3+a_4=0,a_1+a_2+a_3-a_4=0\}$ 或者是 $\{(0,a,-a,0) \mid \alpha\in \mathbf{R}\}$.

                          \item 由维数公式:$\dim V+\dim W=\dim (V\cap W)+\dim (V+W)$,易得$\dim V=3,\enspace \dim W=2,\enspace \dim (V\cap W)=1$,因此$\dim (V+W)=4$,则有$V+W=\mathbf{R}^4$.

                          \item 先得到$W$的基:列出齐次线性方程组:
                                \[\begin{cases}
                                        a_1-a_2-a_3+a_4=0 \\
                                        a_1+a_2+a_3-a_4=0\end{cases}
                                \]
                                高斯消元得到行阶梯矩阵
                                \[\begin{pmatrix}
                                        1 & 0 & 0 & 0  \\
                                        0 & 1 & 1 & -1
                                    \end{pmatrix}\]
                                秩为2,基分别为$\beta_1= (0,-1,1,0)$,$\beta_2= (0,1,0,1)$,则$W$的补空间维数是$\dim \mathbf{R}^4 - \dim W=2$.

                                利用基的扩张求其补空间的基. 也即求$\beta_1,\beta_2,e_1,e_2,e_3,e_4$的极大无关组,其中$e_1,e_2,e_3,e_4$是自然基.
                                \[\begin{pmatrix}
                                        0  & 0 & 1 & 0 & 0 & 0 \\
                                        -1 & 1 & 0 & 1 & 0 & 0 \\
                                        1  & 0 & 0 & 0 & 1 & 0 \\
                                        0  & 1 & 0 & 0 & 0 & 1
                                    \end{pmatrix}\rightarrow
                                    \begin{pmatrix}1 & 0 & 0 & 0 & 1 & 0  \\
                 0 & 1 & 0 & 1 & 1 & 0  \\
                 0 & 0 & 1 & 0 & 0 & 0  \\
                 0 & 0 & 0 & 1 & 1 & -1
                                    \end{pmatrix}\]
                                则$\beta_1,\beta_2,e_1,e_2$即是扩张后的基,因此$W$的补空间的一组基为$e_1,e_2$.
                      \end{enumerate}
            \end{enumerate}
        \end{answer}
        \item 设 $f_1=-1+x,\ f_2=1-x^2,\ f_3=1-x^3,\ g_1=x-x^2,\ g_2=x+x^3,\ V_1=\spa(f_1,f_2,f_3),\ V_2=\spa(g_1,g_2)$,求:
        \begin{enumerate}
            \item $V_1+V_2$ 的基和维数;
            \item $V_1 \cap V_2$ 的基和维数;
            \item $V_2$ 在 $\mathbf{R}[x]_4$ 空间的补.
        \end{enumerate}
        \begin{answer}
            \begin{enumerate}
                \item 先将各向量用坐标表示:
                      \begin{gather*}
                          f_1=(-1,1,0,0),f_2=(1,0,-1,0),f_3=(1,0,0,-1) \\
                          g_1=(0,1,-1,0),g_2=(0,1,0,1) \\
                          V_1+V_2=\spa(f_1,f_2,f_3,g_1,g_2)
                      \end{gather*}
                      只需求 $f_1,f_2,f_3,g_1,g_2$ 的极大线性无关组即可.
                      \[\begin{pmatrix}
                              -1 & 1  & 1  & 0  & 0 \\
                              1  & 0  & 0  & 1  & 1 \\
                              0  & -1 & 0  & -1 & 0 \\
                              0  & 0  & -1 & 0  & 1
                          \end{pmatrix}\rightarrow
                          \begin{pmatrix}1 & -1 & -1 & 0 & 0 \\
                 0 & 1  & 1  & 1 & 1 \\
                 0 & 0  & 1  & 0 & 1 \\
                 0 & 0  & 0  & 0 & 1
                          \end{pmatrix}\]
                      则极大线性无关组为 $f_1,f_2,f_3,g_2$,因此 $V_1+V_2$ 的基为 $f_1,f_2,f_3,g_2$,维数为4.

                \item 易得$\dim V_1=3,\enspace \dim V_2=2$. 则$\dim (V_1\cap V_2)= \dim V_1+\dim V_2-\dim (V_1+V_2)=1$. 只需找到属于$V_1\cap V_2$的一个向量,其就是$V_1\cap V_2$的基.
                      \[U=\lambda_1f_1+\lambda_2f_2+\lambda_3f_3= \mu_1g_1+\mu_2g_2,\]
                      已知$g_1$可被$f_1,f_2,f_3,g_2$表示,则只需取$\mu_1=1$,再求解其它系数即解得:
                      \[\lambda_1=1,\lambda_2=1,\lambda_3=0,\mu_2=0,\]
                      因此$u=g_1=(0,1,-1,0)$或者$x-x^2$是$V_1\cap V_2$的一组基.

                \item 只需求$g_1,g_2,e_1,e_2,e_3,e_4$的极大无关组即可.
                      \[\begin{pmatrix}
                              0  & 0 & 1 & 0 & 0 & 0 \\
                              1  & 1 & 0 & 1 & 0 & 0 \\
                              -1 & 0 & 0 & 0 & 1 & 0 \\
                              0  & 1 & 0 & 0 & 0 & 1\end{pmatrix}\rightarrow
                          \begin{pmatrix}
                              1 & 1 & 0 & 1 & 0 & 0  \\
                              0 & 1 & 0 & 1 & 1 & 0  \\
                              0 & 0 & 1 & 0 & 0 & 0  \\
                              0 & 0 & 0 & 1 & 1 & -1\end{pmatrix}\]
                      则$g_1,g_2,e_1,e_2$是一组极大无关组. $V_2$在$\mathbf{R}[x]_4$的补是$\spa(e_1,e_2)$其中$e_1=1,e_2=x$.
            \end{enumerate}
        \end{answer}
        \item 证明或给出反例:
        \begin{enumerate}
            \item 设 $U_1, U_2, U_3$ 都是 $V$ 的子空间,若 $U_1 + U_3 = U_2 + U_3$,则 $U_1 = U_2$.
            \item 设 $U_1, U_2, U_3$ 都是 $V$ 的子空间,若 $U_1 \oplus U_3 = U_2 \oplus U_3$,则 $U_1 = U_2$.
        \end{enumerate}
        \begin{answer}

        \end{answer}

        \item 若 $V$ 的子空间 $U_1,U_2,U_3$ 满足 $U_1 \cap U_2 = U_2 \cap U_3 = U_3 \cap U_1 = \{0\}$,问 $U_1 + U_2 + U_3$ 是否是直和?
        \begin{answer}

        \end{answer}

        \item 设 $V_1,V_2$ 是线性空间 $V$ 的两个非平凡子空间,证明: $\exists \alpha \in V$,使得$\alpha \notin V_1$ 且 $\alpha \notin V_2$.并在$\mathbf{R}^3$ 中举一例.
        \begin{answer}
            \textbf{证:} 因为 $V_1, V_2$ 是非平凡子空间, 所以存在 $\alpha \notin V_1$. 若 $\alpha \notin V_2$, 则命题得证;若 $\alpha \in V_2$, 另有 $\beta \notin V_2$, 此时若 $\beta \notin V_1$, 命题也得证. 设 $\beta \in V_1$, 则 $\alpha \notin V_1, \alpha \in V_2, \beta \in V_1, \beta \notin V_2$.
            考虑 $\alpha + \beta \in V_1$, 由 $\beta \in V_1 \implies \alpha \in V_1$, 从而得出矛盾, 所以 $\alpha + \beta \notin V_1$. 类似可证 $\alpha + \beta \notin V_2$. 于是命题成立.
        \end{answer}
        \item 设 $V_1,\ldots,V_m (m > 2)$ 是线性空间$V$的$m$个非平凡子空间,证明:$V$ 中存在一个同时不属于任何一个$V_i(1 \leqslant i \leqslant m)$的向量.并在$\mathbf{R}^3$ 中举一例.
        \begin{answer}\label{eg:4:A:5}
          \textbf{证:} (归纳法)当 $r=2$ 时由上题知成立.
          假设 $r = k-1$ 时也成立, 即存在 $\alpha \in V$
          使得 $\alpha$ 同时不属于 $V_i (i=1,2,\ldots,k-1)$
          下面证明 $r=k$ 时也成立. 显然当此 $\alpha \notin V_k$ 则 $\alpha$ 即为所求.
           若 $\alpha \in V_k$, 由于 $V_k$ 是 $V$ 的真子空间, 故有 $\beta \notin V_k$, 易知对任意 $V_i (i=1,2,\ldots,k-1)$ 都至多有一个 $k_i$, 使得 $\beta + k_i \alpha \notin V_i; i=1,2,\ldots,k-1$, 我们取异于 $k_i$ 的数 $k$, 则必有 $\beta + k \alpha \notin V_i; i=1,2,\ldots,k-1$. 又由于 $\alpha \in V_k, \beta \notin V_k$ 故有 $\beta + k \alpha \notin V_k$, 故 $\beta + k \alpha \notin V_k (i = 1,2,\dots,k)$.
        \end{answer}
    \end{exgroup}

    \begin{exgroup}
        \item 已知$V_1$是线性方程组\[\begin{cases}
                3x_1+4x_2-5x_3+7x_4=0 \\
                4x_1+11x_2-13x_3+16x_4=0
            \end{cases}\]
        的解空间,$V_2$是线性方程组\[\begin{cases}
                2x_1-3x_2+3x_3-2x_4=0 \\
                7x_1-2x_2+x_3+3x_4=0
            \end{cases}\]
        的解空间,分别求$V_1 \cap V_2$与$V_1+V_2$的基和维数.
        \begin{answer}
            \begin{enumerate}
                \item $V_1\cap V_2$ 即是两组方程组合并后的解空间.
                      \[\begin{pmatrix}
                              3 & 4  & -5  & 7  \\
                              4 & 11 & -13 & 16 \\
                              2 & -3 & 3   & -2 \\
                              7 & -2 & 1   & 3
                          \end{pmatrix}\rightarrow
                          \begin{pmatrix}1 & 7   & -8 & 9   \\
                 0 & -17 & 19 & -20 \\
                 0 & 0   & 0  & 0   \\
                 0 & 0   & 0  & 0
                          \end{pmatrix}\]
                      则解向量$u_1=\left(\dfrac {3}{17},\dfrac{19}{17},1,0\right)^T,u_2=\left(\dfrac{13}{17},-\dfrac{20}{17},0,1\right)^T$. $u_1,u_2$是$V_1\cap V_2$的基,则其维数为2.

                \item 易得$\dim V_1=2,\dim V_2=2$,则由维数公式$\dim (V_1+V_2)=\dim V_1+\dim V_2-\dim (V_1\cap V_2)= 2$又$V_1\cap V_2\subseteq V_1+V_2$,且二者维数相等. 则$V_1\cap V_2=V_1+V_2$,亦即$V_1=V_2$. 所以$V_1+V_2$的基也是$u_1,u_2$,同$V_1\cap V_2$.
            \end{enumerate}
        \end{answer}
        \item 设$W_1,W_2$是线性空间$V(\mathbf{F})$的两个子空间. 证明以下命题等价:
        \begin{enumerate}
            \item $W_1 \cup W_2$为$V$ 的子空间;

            \item $W_1 \subseteq W_2$或$W_2 \subseteq W_1$;

            \item $W_1 \cup W_2=W_1+W_2$.
        \end{enumerate}

        \begin{answer}
            首先证明 (1) (2) 等价. 显然 (2) 可以直接推出 (1),下证 (1) 可推出 (2) 成立. 反证法:若存在$\alpha\in W_1$,有$\alpha\not\in W_2$,并且存在$\beta \in W_2$,有$\beta \not\in W_1$. 则$\alpha\in W_1\cup W_2$,且$\beta \in W_1\cup W_2$. 但$\alpha+\beta \not\in W_1$且$\alpha+\beta \not\in W_2$,即$\alpha+\beta \not\in W_1\cup W_2$. 这与$W_1\cup W_2$是子空间矛盾. 因此任意$\alpha\in W_1$,有$\alpha\not\in W_2$,或者任意$\beta \in W_2$,有$\beta\not\in W_1$. 即$W_1\subseteq W_2$或者$W_2\subseteq W_1$.

          接下来证明 (1) (3) 等价:显然 (3) 可以直接推出 (1),下证 (1) 推出 (3) :因为$W_1\cup W_2$是$V$的子空间,对于$\forall\alpha\in W_1,\forall	\beta \in W_2,\alpha\in W_1\cup W_2$且$\beta \in W1\cup W_2$,则 $\lambda\alpha+\mu\beta\in W_1\cup W_2$. 这与和空间的定义是完全一致的. 因此,$W_1\cup W_2=W_1+W_2$ 得证. 综上,以上三命题是等价的.

          (2) (3) 的等价性留待读者自行验证,与证明 (1) (2) 等价是基本一致的.
        \end{answer}

        \item 证明:每个 $n$ 维空间均可以表示为 $n$ 个一维子空间的直和.
        \begin{answer}
            我们很自然地想到,可以任取 $n$ 维空间 $V$ 的一组基 $\alpha_1,\alpha_2,\ldots,\alpha_n$,然后取 $V_i = \spa(\alpha_i)$. 如果 $V = V_1 \oplus V_2 \oplus \cdots \oplus V_n$,我们就成功地为每个 $n$ 维空间找到了写成 $n$ 个一维子空间直和的分解形式. 下面我们来证明这一构造的正确性:
            \begin{enumerate}
                \item
                \item
            \end{enumerate}
        \end{answer}

        \item 设$V$是域$\mathbf{F}$上的$n$维线性空间,$\alpha_1,\alpha_2,\ldots,\alpha_n$是$V$的一组基,且
        \begin{gather*}
            V_1=\spa(\alpha_1+2\alpha_2+\cdots+n\alpha_n) \\
            V_2=\left\{k_1\alpha_1+k_2\alpha_2+\cdots+k_n\alpha_n \;\middle|\; k_1+\dfrac{k_2}{2}+\cdots+\dfrac{k_n}{n}=0\right\}
        \end{gather*}
        证明:
        \begin{enumerate}
            \item $V_2$是$V$的子空间;

            \item $V=V_1\oplus V_2$.
        \end{enumerate}

        \begin{answer}
            \begin{enumerate}
                \item 显然$V_2\subseteq V$,只需证明$V_2$封闭即可. 对于$v=k_1\alpha_1+\cdots+k_n\alpha_n,v'=k_1'\alpha_1+\cdots+k_n'\alpha_n$. $\lambda v+\mu v'=(\lambda k_1+\mu k_1')\alpha_1+\cdots+(\lambda k_n+\mu k'_n)\alpha_n$,因此  $\lambda v+\mu v'\in V_2\in V_2$,$V_2$ 封闭,则$V_2$是$V$的子空间.

                \item 设$W=V_1+V_2$. 对任意 $v\in V_1,v=\lambda \alpha_1+2\lambda\alpha_2+\cdots+n\lambda\alpha_n$,则 $\lambda+2\dfrac{\lambda}2+\allowbreak\cdots+n\dfrac{\lambda}n=n\lambda$,因此当$v\ne 0$时,$v\not\in v_2$,则 $V_1\cap V_2=\{0\}$,有$W=V_1\oplus V_2$.

                      下证$W=V$,易得$\dim V_1=1$. 对于$V_2$,写成坐标形式,则其是一个方程组的解空间,系数矩阵 $A=\begin{pmatrix}
                              1 & \frac 12 & \cdots & \frac 1n \\
                              0 &          &        &          \\
                                & \ddots   &        &          \\
                                &          & \ddots &          \\
                                &          &        & 0
                          \end{pmatrix}$. 方程组为 $AK=0$,则 $\dim V_2=n-r(A)=n-1$,因此$\dim W=\dim V_1+\dim V_2=n=\dim V$,又$W\subseteq V$,则$W=V$得证,综上$V=V_1\oplus V_2$.
            \end{enumerate}
        \end{answer}

        \item 完成以下三个问题:
        \begin{enumerate}
            \item 设 $U = \{(x,x,y,y) \mid x,y\in \mathbf{F}\}$,求 $\mathbf{F}^4$ 的一个子空间 $W$,使得 $\mathbf{F}^4 = U \oplus W$.
            \item 设 $U = \{(x,y,x+y,x-y,2x) \mid x,y\in \mathbf{F}\}$,求 $\mathbf{F}^5$ 的一个子空间 $W$,使得 $\mathbf{F}^5 = U \oplus W$.
            \item 设 $U = \{(x,y,x+y,x-y,2x) \mid x,y\in \mathbf{F}\}$,求 $\mathbf{F}^5$ 的三个都不为 $\{\vec{0}\}$ 的子空间 $W_1,W_2,W_3$,使得 $\mathbf{F}^5 = U \oplus W_1 \oplus W_2 \oplus W_3$.
        \end{enumerate}
        \begin{answer}

        \end{answer}

        \item 已知$V_1,V_2$是有限维线性空间$V$的子空间,且$\dim(V_1+V_2)=\dim(V_1 \cap V_2)+1$. 证明:要么$V_1 \subseteq V_2$,要么$V_2 \subseteq V_1$.

        \begin{answer}
            \begin{enumerate}
                \item 法一:反证法. 如果$\exists v_1\in V_1,v_1\not\in V_2$,且 $\exists v_2\in V_2,v_2\not\in V_1$,则有 $v_1,v_2\not \in V_1\cap V_2$ 并且 $v_1,v_2$ 线性无关. 设 $A=\{\alpha_1,\ldots ,\alpha_n\}$是 $V_1\cap V_2$的基. 由基扩张,设$B=\{\alpha_1,\ldots ,\alpha_n,\alpha_{n+1},\ldots ,\alpha_{n+k}\}$是$V_1+V_2$的基. 由于$v_1,v_2$不在$V_1\cap V_2$中,因此$A$无法表出$v_1,v_2$. 则$\alpha_1,\ldots ,\alpha_n,v_1,v_2$ 线性无关. 又$v_1,v_2\in V_1+V_2$,则$\alpha_1,\ldots ,\alpha_n,v_1,v_2$可由$\alpha_1,\ldots ,\alpha_{n+k}$ 线性表出. 由定理 $3.3$ 得 $n+2\leqslant n+k$,即 $\dim (V_1+V_2)\ge\dim (V_1\cap V_2)+2$,这与条件矛盾. 因此,$\forall v_1\in V_1,v_1\in V_2$ 或 $\forall v_2\in V_2,v_2\in V_1$. 即 $V_1\subseteq V_2$ 或 $V_2\subseteq V_1$,得证.

                \item 法二:维数公式. $\dim V_1+\dim V_2=\dim  (V_1+V_2)+\dim  (V_1\cap V_2)=2\dim  (V_1\cap V_2)+1$,即 $(\dim V_1-\dim  (V_1\cap V_2))+ (\dim V_2-\dim  (V_1\cap V_2))=1$. 于是,要么 $\dim V_1-\dim  (V_1\cap V_2)=0$,要么 $\dim V_2-\dim  (V_1\cap V_2)=0$,即 $V_1\subseteq V_2$ 或 $V_2\subseteq V_1$,得证.
            \end{enumerate}
        \end{answer}

        \item 判断下列说法是否正确:
        \begin{enumerate}
            \item 若$V \subseteq V_1 \cup V_2 \cup \cdots \cup V_s$,则$V=(V_1 \cap V)\cup(V_2 \cap V)\cup\cdots\cup(V_s \cap V)$;

            \item 若$V \subseteq V_1+V_2+\cdots +V_s$,则$V=(V_1 \cap V)+(V_2 \cap V)+\cdots+(V_s \cap V)$.
        \end{enumerate}
        \begin{answer}
          \begin{enumerate}
              \item 正确,若$\alpha\in V$,则$\alpha \in V_1\cup V_2\cdots \cup V_s$,即$\alpha \in V\cap (V_1\cup V_2\cdots \cup V_s)$,则$\alpha \in (V\cap V_1)\cup \cdots \cup (V\cap V_s)$,则$V\subseteq(V_1\cap V)\cup \cdots \cup (V_s\cap V)$. 另一边类似的同样成立,则有 $V=(V_1\cap V)\cup \cdots \cup (V_s\cap V)$ 得证.

              \item 错误,反例:设 $V_1,V_2,V_3$ 是平面$K$上三条过原点$O$的不重合直线,则$V\subseteq K=V_1+V_2$,但$V\cap V_1=\{0\},V\cap V_2=\{0\}$,$V\ne (V\cap V_1)+(V\cap V_2)$.
          \end{enumerate}
        \end{answer}
        \item 设$V$为有限维线性空间,$V_1$为其非零子空间. 证明:存在唯一的子空间$V_2$,使得$V=V_1\oplus V_2$的充要条件为$V_1=V$.
        \begin{answer}

        \end{answer}

       \item 设$A_1$和$A_2$均为$V$的仿射子集,证明:$A_1\cap A_2$是$V$的仿射子集或空集(可推广至任意交).
       \begin{answer}
        设$A,B$为$V$的仿射子集,若$A,B$交于一点$v$,则令$\{0\}$为子空间,此时$A \cap B = v+\{0\} $,仍然是子空间,若至少有两点$u,v \in A \cap B$,则$ \forall u,v \in A \cap B$,我们
        \begin{align*}
         \lambda u + (1-\lambda)v \in A \\
            \lambda u + (1-\lambda)v \in B\\
            \lambda u + (1-\lambda)v \in A \cap B
        \end{align*}
       故A,B的交集是仿射子集.
       空集的情况显然不必多说.
       \end{answer}
    \end{exgroup}

    \begin{exgroup}
        \item 设$W_0,W_1,W_2,\ldots,W_s$是线性空间$V$的$s+1$个非平凡子空间,且$W_0 \subseteq W_1 \cup W_2 \cup \cdots \cup W_s$. 证明:必存在$i$使得$W_0\subseteq W_i$.
        \begin{answer}
            由前 $B$ 组第 8 题的证明可知 $W_0=(W_1\cap W_2)\cup\cdots\cup(W_s\cap W_0)$. 由于$W_1\cap W_2\cdots W_s\cap W_0$都是$W_0$的子空间,根据覆盖定理,必存在$i$,使得$W_0=W_i\cap W_0$,即 $W_0\subseteq W_i$ 得证.
        \end{answer}

        \item 受三个有限集之并集的元素数量公式的启发,你可能会这样猜测:如果 $V_1,V_2,V_3$ 是一有限维向量空间的子空间,那么有
        \begin{align*}
            \dim(V_1+V_2+V_3) =\ &\dim V_1+\dim V_2+\dim V_3 \\
            &-\dim(V_1\cap V_2)-\dim(V_1\cap V_3)-\dim(V_2\cap V_3) \\
            &+\dim(V_1\cap V_2\cap V_3).
        \end{align*}
        解释一下为什么这样猜测,然后证明以上公式或给出一反例.
        \begin{answer}

        \end{answer}

        \item 证明:如果 $V_1,V_2,V_3$ 是一有限维向量空间的子空间,那么
        \begin{align*}
            &\dim(V_1+V_2+V_3) \\
            &= \dim V_1+\dim V_2+\dim V_3 \\
            &-\dfrac{\dim(V_1\cap V_2)+\dim(V_1\cap V_3)+\dim(V_2\cap V_3)}{3} \\
            &+\dfrac{\dim((V_1 + V_2) \cap V_3) - \dim((V_1 + V_3) \cap V_2) + \dim((V_2 + V_3) \cap V_1)}{3}.
        \end{align*}

        注:以上公式可能显得有点奇怪,因为右边看着不像一个整数.
        \begin{answer}

        \end{answer}

        \item 设$v_1,\ldots,v_m\in V$. 令
        \[A=\{\lambda_1v_1+\cdots+\lambda_mv_m \mid \lambda_1,\ldots,\lambda_m\in\mathbf{F}\text{~且~}\lambda_1+\cdots+\lambda_m=1\}.\]
        证明:
        \begin{enumerate}
            \item $A$是$V$的仿射子集;

            \item $V$的每个包含$v_1,\ldots,v_m$的仿射子集均包含$A$;

            \item 存在某个$v\in V$和$V$的子空间$U$使得$A=v+U$且$\dim U\leqslant m-1$.
        \end{enumerate}
    \begin{answer}
        \begin{enumerate}
            \item 对于 $v=\lambda_1 v_1+\cdots+\lambda_m v_m\in A$ 和 $w=\eta_1 v_1+\cdots+\eta_m v_m\in A$, 且 $\lambda_1,\cdots,\lambda_m\in  \mathbf{F}$, $\lambda_1+\cdots+\lambda_m=1$ , $\eta_1,\cdots,\eta_m\in \mathbf{F}$, $\eta_1+\cdots+\eta_m=1$. 对于任意的$\lambda\in \mathbf{F}$, 我们有
            \begin{align*}
                \lambda v+(1-\lambda) w=\sum_{i=1}^m(\lambda\lambda_i+(1-\lambda)\eta_i)v_i.
            \end{align*}
            注意到
            \[ \sum_{i=1}^m(\lambda\lambda_i+(1-\lambda)\eta_i)=\lambda\sum_{i=1}^m\lambda_i+(1-\lambda)\sum_{i=1}^n\eta_i=\lambda+(1-\lambda)=1, \]
            我们有$\lambda v+(1-\lambda) w\in A$. 故 $A$ 是 $V$ 的一个仿射子集
        \item 我们使用数学归纳法来证明,对于任意 $V$ 的仿射子集 $W$ ,其包含$v_1$, $\cdots$, $v_m$$k\le m$,
            如果 $\lambda_1+\cdots+\lambda_k=1$,
            我们有 \[ \sum_{j=1}^k\lambda_jv_j\in W. \]

            当$k=1$ 和 $k=2$ 时,结论是显然的.
            假设 $k$ 时成立,
            接下来对于 $k+1$ ($k+1\leqslant m$).
            我们假设$\lambda_1+\cdots+\lambda_{k+1}=1$.
            若 $\lambda_{k+1}=1$, 那么\[\sum_{j=1}^{k+1}\lambda_jv_j=v_{k+1}\in W.\]
            若$\lambda_{k+1}\ne 1$,那么\[ \frac{1}{1-\lambda_{k+1}}(\lambda_1+\cdots+\lambda_k)=1. \]
            由归纳假设,我们有\[ \frac{1}{1-\lambda_{k+1}}(\lambda_1v_1+\cdots+\lambda_kv_k)\in W. \]
            也有 \[ (1-a_{k+1})\left(\frac{1}{1-\lambda_{k+1}}(\lambda_1v_1+\cdots+\lambda_kv_k)\right)+a_{k+1}v_{k+1}\in W, \]
            即 \[ \lambda_1v_1+\cdots+\lambda_{k+1}v_{k+1}\in W. \]
            故而 $\lambda_1,\cdots,\lambda_m\in  \mathbf{F}$, $\lambda_1+\cdots+\lambda_m=1$,
            \[\lambda_1 v_1+\cdots+\lambda_m v_m\in W,\]有$A\subset W$.
        \item  注意到$\lambda_1+\cdots+\lambda_m=1$, 我们有\[ \lambda_1 v_1+\cdots+\lambda_m v_m=v_1+\lambda_2(v_2-v_1)+\cdots+\lambda_m(v_m-v_1). \]所以 $A\subset v_1+\spa(v_2-v_1,\cdots,v_m-v_1)$.
            类似的,对于任意\[v\in v_1+\spa(v_2-v_1,\cdots,v_m-v_1),\]$v$ 可以被写为 \[ v_1+\sum_{i=2}^m \lambda_i(v_i-v_1)=(1-\lambda_2-\cdots-\lambda_m)v_1+\sum_{i=2}^m \lambda_i v_i \]
            对于 $\lambda_2$, $\cdots$, $\lambda_m\in \mathbf{F}$. 注意到\[ (1-\lambda_2-\cdots-\lambda_m)+\sum_{i=2}^m \lambda_i=1, \]
            可以推出 $v_1+\spa(v_2-v_1,\cdots,v_m-v_1)\subset A$.
            因此\[A=v_1+\spa(v_2-v_1,\cdots,v_m-v_1).\]
            $v=v_1,U=\spa(v_2-v_1,\cdots,v_m-v_1)$, 有 $\dim U\le m-1$
        \end{enumerate}
    \end{answer}
            \item (加强的覆盖定理) 设 $A_1, A_2, \ldots, A_n$ 是域 $\mathbf{F}$ 上向量空间 $V$ 的仿射子集,证明 $A_1\cup A_2\cup \cdots \cup A_n$ 不能覆盖 $V$,即存在 $\alpha \in V$ 但是 $\forall 1\leqslant i\leqslant n, \alpha \notin A_n$
    \end{exgroup}
\end{exercise}
